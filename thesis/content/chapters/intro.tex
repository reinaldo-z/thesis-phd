%!TEX root = ../../thesis.tex
\chapter{Introduction}\label{chap:introduction}
\partialtoc

\section{Nonlinear Optics}
\label{sec:nonlinear_optics}

Nonlinear optics studies the modifications of the optical properties of a
material system as a result of its interaction with the electric field of
light. Typically, only laser light is intense enough to produce a modification
in the optical properties of a material system\footnote{Some nonlinear
responses, like luminescence of dye molecules, were discovered previous to the
development of the laser}.
% 
The nonlinearity of those phenomena came from the fact that their manifestation
do not depend in a linear manner on the strength of the optical applied field.

In 1917, Albert Einstein published the theoretical possibility of amplification
of radiation through stimulated emission with coherence \cite{rawicz08PSPIE}
and Valentin A. Fabrikat made the calculation of the conditions needed to
produce radiation amplification by stimulated emission like the population
inversion; he also proposed the resonant cavity \cite{lukishova10JEOSRP}. This
theory and calculations was not carried out to practice until 1952 when Joseph
Weber started the development of masers but the first one was demonstrated by
Charles H. Townes and his working team in 1953. The principal difficulty in the
development of an \emph{optical maser}, was the transition from centimeter to
nanometer waves. In 1960 Theodore H. Maiman presented the first functional ruby
laser \cite{maiman60Nat}.
One year later, second-harmonic generation (SHG) was the the first nonlinear
effect observed by Peter Franken, et al. at University of Michigan
\cite{franken61PRL}. Exciting a piece of crystalline quartz with a laser beam,
they produced a second beam with the double of frequency than the original but
with less intensity than the original one.
% 
In 1962, Nicolaas Bloembergen et al. described the induced nonlinear electric
dipole and higher moments in an atomic system using quantum-mechanical
perturbation theory \cite{armstrong62PR}, and solved the  Maxwell's equations
in nonlinear dielectrics satisfying the boundary conditions at a plane
interface between a linear and nonlinear medium \cite{bloembergen1962PR}.
% 
The development of the first pulsed laser in 1964 by William B. Bridges
\cite{bridges1964APL}, and the later works in this field, allowed the increase
of laser intensities and then the induction of new nonlinear effects in
nonlinear media. 
% 
In 1988, Yuen-Ron Shen et al. studied the the bulk contribution to surface
second-harmonic generation and presented a model for the bulk nonlinearity of
molecular systems including the quadrupole bulk contribution \cite{guyot88PRB}.
% 
Currently, chirped pulse amplification, developed by Gerard Morou and Donna
Strickland \cite{strickland85OC}, in Ti:Saphire lasers became the standard for
high energy, ultrashort pulsed lasers.

\subsection{Nonlinear polarization}
\label{sec:nonlinear_polarization}

To describe the optical nonlinearity, consider how the polarization of a
material $P (t)$, defined as the dipole moment per unit volume, depends on the
strength of an applied optical field $E (t)$. In the case of linear optics, the
polarization dependence is lineal with respect to the strength of electric
field and can be described as
\begin{equation}
P (t) = \epsilon_{0} \chi^{(1)} E (t),
\label{eq:linearpol}
\end{equation}
where $\epsilon_{0}$ is the permittivity of free space and $\chi^{(1)}$ is the
linear susceptibility. In nonlinear optics, the optical response of Eq.
\eqref{eq:linearpol} is characterized by rewriting the polarization as a power
series in the field strength as 
\begin{equation}
\begin{aligned}
P 
=& 
\epsilon_{0} [ \chi^{(1)} E (t) + \chi^{(2)} E^{2} (t) + \chi^{(3)}
E^{3} (t) + \cdots ]
\\
=&
P^{(1)} (t) + P^{(2)} (t) + P^{(3)}
(t) + \cdots
.
\label{eq:polarizations}
\end{aligned}
\end{equation}
The $\chi^{(n)}$ expression is known as the $n^{\text{th}}$ nonlinear optical
susceptibility; the common linear effects presented in most media are described
only by $P^{(1)} (t)$ that reduces the expression \eqref{eq:polarizations} to
\eqref{eq:linearpol}. Considering that the electric field $E$ written here as a
scalar must be considered a vector, then $\chi^{(1)}$ must be a second-rank
tensor, $\chi^{(2)}$ a third-rank tensor and so on, and they depend on the
frequencies of the applied fields \cite{boyd03nonlinear}.

The nonlinear polarization can also be described threating the incident
electric field as a superposition of plane waves. If we assume that the
electric field is written as \cite{anderson16theoretical}
\begin{equation}
\mathbf{E} (\mathbf{r},t) = 
\sum_{n} \mathbf{E}_{n} (\mathbf{r},t)
\label{eq:planar1}
\end{equation}
where
\begin{equation}
\mathbf{E}_{n} (\mathbf{r},t) = 
\mathbf{E}_{n} (\mathbf{r}) e^{-i\omega_{n} t} +  c.c.
,
\label{eq:planar2}
\end{equation}
and the, from Eq. \eqref{eq:polarizations} we can write 
\begin{equation}
\mathbf{P} (\mathbf{r},t) =
\sum_{n} \mathbf{P} (\omega_{n}) e^{-i\omega_{n}t}
.
\label{eq:polarizations}
\end{equation}
Then, we can define the second order nonlinear polarization, that describes the
second order effects in which we have interest, in terms of the second order
susceptibility as 
\begin{equation}
P^{a} (\omega_{n} + \omega_m) =
\epsilon_{0} \sum_{\mathrm{bc}} \sum_{(nm)} 
\mathbf{\chi}^{(2), \mathrm{abc}}
\left(
-(\omega_{n} + \omega_{m}); \omega_{n} + \omega_{m}
\right)
E^{\mathrm{b}} (\omega_{n}) E^{\mathrm{c}} (\omega_{m})
,
\end{equation}
where the roman superscripts $\mathrm{abc}$ denote to Cartesian components of
the fields and $(nm)$ indicate that $n$ and $m$ can be varied while the sum
$\omega_{n} + \omega_{m}$ remains fixed. Making the analysis when we have two
incoming scalar fields with frequencies $\omega_{1}$ and $\omega{2}$
\begin{equation}
E(t) = E_{1} e^{-i\omega_{1}t} + E_{2} e^{-i\omega_{2}t} + c.c.
, 
\label{eq:two-fields}
\end{equation}
we have from Eq. \eqref{eq:polarizations} 


\stopcontents[chapters]










