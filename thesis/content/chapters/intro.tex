%!TEX root = ../../thesis.tex
\chapter{Introduction}\label{chap:introduction}
\partialtoc

\section{Nonlinear Optics}
\label{sec:nonlinear_optics}

Nonlinear optics studies the modifications of the optical properties of a
material system as a result of its interaction with the electric field of
light. Typically, only laser light is intense enough to produce a modification
in the optical properties of a material system\footnote{Some nonlinear
responses, like luminescence of dye molecules, were discovered previous to the
development of the laser.}. The nonlinearity of those phenomena came from the
fact that their manifestation do not depend in a linear manner on the strength
of the optical applied field. The field of nonlinear optics now extends from
fundamental studies of the interaction of light with matter to applications
such as laser frequency conversion, optical switching, and optical tests.

In 1917, Albert Einstein published the theoretical possibility of amplification
of radiation through stimulated emission with coherence \cite{rawicz08PSPIE}
and Valentin A. Fabrikat made the calculation of the conditions needed to
produce radiation amplification by stimulated emission like the population
inversion; he also proposed the resonant cavity \cite{lukishova10JEOSRP}. This
theory and calculations was not carried out to practice until 1952 when Joseph
Weber started the development of masers but the first one was demonstrated by
Charles H. Townes and his working team in 1953. The principal difficulty in the
development of an \emph{optical maser}, was the transition from centimeter to
nanometer waves. In 1960 Theodore H. Maiman presented the first functional ruby
laser \cite{maiman60Nat}.
One year later, second-harmonic generation (SHG) was the the first nonlinear
effect observed by Peter Franken, et al. at University of Michigan
\cite{franken61PRL}. Exciting a piece of crystalline quartz with a laser beam,
they produced a second beam with the double of frequency than the original but
with less intensity than the original one.
% 
In 1962, Nicolaas Bloembergen et al. described the induced nonlinear electric
dipole and higher moments in an atomic system using quantum-mechanical
perturbation theory \cite{armstrong62PR}, and solved the  Maxwell's equations
in nonlinear dielectrics satisfying the boundary conditions at a plane
interface between a linear and nonlinear medium \cite{bloembergen1962PR}.
% 
The development of the first pulsed laser in 1964 by William B. Bridges
\cite{bridges1964APL}, and the later works in this field, allowed the increase
of laser intensities and then the induction of new nonlinear effects in
nonlinear media. 
% 
In 1988, Yuen-Ron Shen et al. studied the the bulk contribution to surface
second-harmonic generation and presented a model for the bulk nonlinearity of
molecular systems including the quadrupole bulk contribution \cite{guyot88PRB}.
% 
Currently, chirped pulse amplification, developed by Gerard Morou and Donna
Strickland \cite{strickland85OC}, in Ti:Saphire lasers became the standard for
high energy, ultrashort pulsed lasers.

\subsection{Nonlinear polarization}
\label{sec:nonlinear_polarization}

To describe the optical nonlinearity we need to consider how the polarization
of a material $P(\omega)$, defined as the dipole moment per unit volume,
depends on the strength of an applied optical field $E (t)$. In the case of
linear optics, the polarization dependence is lineal with respect to the
strength of electric field and can be described as
\begin{equation}
P (\omega) = \epsilon_{0} \chi^{(1)} E (\omega),
\label{eq:linearpol}
\end{equation}
where $\epsilon_{0}$ is the permittivity of free space and $\chi^{(1)}$ is the
linear susceptibility. In nonlinear optics, the optical response of Eq.
\eqref{eq:linearpol} is characterized by rewriting the polarization as a power
series in the field strength as 
\begin{equation}
\begin{aligned}
P (\omega)
=& 
\epsilon_{0} [ 
\chi^{(1)} E (\omega) + 
\chi^{(2)} E^{2} (\omega) + 
\chi^{(3)} E^{3} (\omega) + 
\cdots ]
\\
=&
P^{(1)} (\omega) + P^{(2)} (\omega) + P^{(3)} (\omega) + \cdots
.
\label{eq:polarizations}
\end{aligned}
\end{equation}
The $\chi^{(n)}$ expression is known as the $n^{\text{th}}$ nonlinear optical
susceptibility; the common linear effects presented in most media are described
only by $P^{(1)} (\omega)$ that reduces the expression \eqref{eq:polarizations}
to\eqref{eq:linearpol}. 
% 
The reason why the polarization has a determining role in the characterization
of nonlinear optical phenomena comes from the fact that time-varying
polarization can produce new components of the electromagnetic field. This can
be shown using the wave equation in nonlinear media that is written as
\cite{boyd03book}
\begin{equation}
\nabla^{2} E - \frac{n^2}{c^2} \frac{\partial^{2} E}{\partial t^{2}} = 
\frac{1}{\epsilon_{0} c^{2}}\frac{\partial^{2} P^{NL}}{\partial t^{2}}
,
\label{eq:waveeq}
\end{equation}
where $n$ is the usual linear refractive index and $c$ is the speed of light in
vacuum. This is an inhomogeneous wave equation in which the polarization $P^
{NL}$ related with the nonlinear response depends on the electric field $E$.
The expression $\partial^{2}P^{NL}/\partial t^{2}$ is a measure of the
acceleration of the charges in the media that which generate new
electromagnetic radiation.

Considering that the electric field $E$ written here as a scalar must be
considered a vector, then $\chi^{(1)}$ must be a second-rank tensor,
$\chi^{(2)}$ a third-rank tensor and so on, and they depend on the frequencies
of the applied fields \cite{boyd03book}. The nonlinear polarization can also be
described threating the incident electric field as a superposition of plane
waves. If we assume that the electric field is written as \cite{janner98PHDT}
\begin{equation}
\mathbf{E} (\mathbf{r},\omega) = 
\sum_{n} \mathbf{E}_{n} (\mathbf{r},\omega)
\label{eq:planar1}
\end{equation}
where
\begin{equation}
\mathbf{E}_{n} (\mathbf{r},\omega) = 
\mathbf{E}_{n} (\mathbf{r}) e^{-i\omega_{n} t} +  c.c.
,
\label{eq:planar2}
\end{equation}
and then, from Eq. \eqref{eq:polarizations} we can write 
\begin{equation}
\mathbf{P} (\mathbf{r},\omega) =
\sum_{n} \mathbf{P}^{(n)} (\omega_{n}) e^{-i\omega_{n}t}
.
\label{eq:vec-polarization}
\end{equation}
Then, we can define the $n^{\mathrm{th}}$ order nonlinear polarization in terms
of the $n^{\mathrm{th}}$ order susceptibility as
\begin{equation}
\begin{aligned}
\mathbf{P}^{(n)} & (\mathbf{r},\Omega) = 
&\epsilon_{0} \sum_{\mathrm{bc}} \sum_{\Omega} 
\boldsymbol{\chi}^{(n)}
\left(
-\Omega; \omega_{1}, \omega_{2}, \ldots, \omega_{n}
\right)
\mathbf{E}(\mathbf{r},\omega_{1}) \mathbf{E}(\mathbf{r},\omega_{2}) (\cdots) 
\mathbf{E}(\mathbf{r},\omega_{n})
,
\end{aligned}
\end{equation}
where $\Omega = \omega_{1} + \omega_{2} + \ldots + \omega_{n}$. Making emphasys
about we are interested in second order effects, we can analyze the case when
we have two incoming fields, with frequencies $\omega_ {1}$ and
$\omega_{2}$, and write the total incident electric field as
\cite{anderson16PHDT}
\begin{equation}
\mathbf{E}(\mathbf{r},\omega) = 
\mathbf{E}_{1}(\mathbf{r}) e^{-i\omega_{1}t} + 
\mathbf{E}_{2}(\mathbf{r}) e^{-i\omega_{2}t} + c.c.
. 
\label{eq:two-fields}
\end{equation}
Then, from Eq. \eqref{eq:polarizations}, the second order polarization can be
written as
\begin{equation}
\mathbf{P}^{(2)} (\mathbf{r};\omega) = 
\epsilon_{0} \boldsymbol{\chi}^{(2)} [\mathbf{E}(\mathbf{r},\omega)]^{2}
,
\label{eq:second-pol-1}
\end{equation}
and using Eq. \eqref{eq:two-fields} and omitting the $(\mathbf{r},\omega)$
dependence in the electric fields we can write
\begin{equation}
\begin{aligned}
\mathbf{P}^{(2)} (\mathbf{r};\omega) 
= &\,\,
\epsilon_{0} \boldsymbol{\chi}^{(2)}
[ \mathbf{E}_{1}^2 e^{-i2\omega_{1}t} + 
\mathbf{E}_{2}^{2} e^{-i2\omega_{2}t} 
\\
&+
2\mathbf{E}_{1}\mathbf{E}_{2} e^{-i(\omega_{1} + \omega_{2})t} + 
2\mathbf{E}_{1}\mathbf{E}_{2}^{*} e^{-i(\omega_{1} - \omega_{2})t} + c.c. ] \\
&+
2 \epsilon_{0} \boldsymbol{\chi}^{(2)} 
[ \mathbf{E}_{1}\mathbf{E}_{1}^{*} + \mathbf{E}_{2}\mathbf{E}_{2}^{*} ]
.
\end{aligned}
\end{equation}
Separating this expression in its components and the nonlinear effect produced
we have
\begin{subequations}
\begin{align}
\mathbf{P}(\mathbf{r};2\omega_{1}) &= \epsilon_{0} 
\boldsymbol{\chi}^{(2)} \mathbf{E}_{1}^{2} e^{-i2\omega_{1}t} +c.c., 
\label{eq:shg-1}
\\
\mathbf{P}(\mathbf{r};2\omega_{1}) &= \epsilon_{0} 
\boldsymbol{\chi}^{(2)} \mathbf{E}_{2}^{2} e^{-i2\omega_{2}t} +c.c.,
\label{eq:shg2}
\\
\mathbf{P}(\mathbf{r};\omega_{1} + \omega_{2}) &= 2\epsilon_{0}
\boldsymbol{\chi}^{(2)} \mathbf{E}_{1} 
\mathbf{E}_{2} e^{-i(\omega_{1}+\omega_{2})t} +c.c, 
\label{eq:sfg}
\\
\mathbf{P}(\mathbf{r};\omega_{1} - \omega_{2}) &= 2\epsilon_{0}
\boldsymbol{\chi}^{(2)} \mathbf{E}_{1} 
\mathbf{E}_{2}^{*} e^{-i(\omega_{1}+\omega_{2})t} +c.c.,
\label{eq:dfg}
\\
\mathbf{P}(\mathbf{r};0) &= 2\epsilon_{0}
\boldsymbol{\chi}^{(2)} (\mathbf{E}_{1}\mathbf{E}_{1}^{*} + 
\mathbf{E}_{2}\mathbf{E}_{2}^{*}) + c.c..
\label{eq:or}
\end{align}
\label{eq:second-responses}
\end{subequations}
The expressions of Eq. \eqref{eq:second-responses} correspond to the following
second order optical processes of light interacting with a nonlinear media:
\begin{itemize}\itemsep0pt
    \item Second harmonic generation (SHG, Eqns. \ref{eq:shg-1} and
    \ref{eq:shg2}) is a process in which photons with same frequency produce
    new photons with twice the energy and frequency and half the wavelength,
    respective to the initial photons.
    \item Sum frequency generation (SFG, Eq. \ref{eq:sfg}) is a process in
    which photons with different frequency generate photons with frequency
    equal to the sum of the frequencies of the originals.
    \item Difference frequency generation (DFG, Eq. \ref{eq:dfg}) is a
    process in which the frequency of the generated photons is the difference
    between two other photon frequencies.
    \item Optical rectification (OR, Eq. \ref{eq:or}) is a process that
    consists of the generation of a quasi-DC polarization.
\end{itemize}
In table \ref{tab:optical-processes} are presented the most common optical
processes of low order \cite{janner98PHDT}.

\begin{table}[tb]
\caption{Optical processes described with $\boldsymbol{\chi}^{(n)}$ .
\label{tab:optical-processes}}
\centering
\scalebox{0.9}{
\begin{tabular}{ c c c p{9cm} c}
\hline
\hline 
\multicolumn{3}{c}
{\,\,\,
$\boldsymbol{\chi}^{(n)}(-(\omega_{1} + \ldots + \omega_{n});\omega_{1},\ldots,
\omega_{n})$}
& \textbf{Process} & \textbf{Order} \\
\hline 
$-\omega_{1}$ & ; & $\omega_{1}$
& Linear absorption / emission and refractive index & 1 \\
\hline
$0$ & ; & $\omega_{1},-\omega_{1}$
& Optical rectification (OR) & 2 \\
$-\omega_{1}$ & ; & $0,\omega_{1}$
& Pockels effect & 2 \\
$-2\omega_{1}$ & ; & $\omega_{1},\omega_{1}$
& Second-harmonic generation (SHG) & 2 \\
$-(\omega_{1}+\omega_{2})$ & ; & $\omega_{1},\omega_{2}$
& Sum-frequency generation (SFG) & 2 \\
$-(\omega_{1}-\omega_{2})$ & ; & $\omega_{1},\omega_{2}$
& Difference-frequency generation (DFG) / 
Parametric amplification and oscillation & 2 \\
\hline
$-\omega_{1}$ & ; & $0,0,\omega_{1}$
& d.c. Kerr effect & 3 \\
$-2\omega_{1}$ & ; & $0,\omega_{1},\omega_{1}$
& Electric Field induced SHG (EFISH) & 3 \\
$-3\omega_{1}$ & ; & $\omega_{1},\omega_{1},\omega_{1}$
& Third-harmonic generation (THG) & 3 \\
$-\omega_{1}$ & ; & $\omega_{1},-\omega_{1},\omega_{1}$
& Degenerate four-wave mixing (DFWM) & 3 \\
$-\omega_{1}$ & ; & $-\omega_{2},\omega_{2},\omega_{1}$
& Two-photon absorption (TPA) / ionization / emission & 3 \\ 
\hline
\hline
\end{tabular}}
\end{table}

\section{Symmetry consequences in nonlinear media}
\label{sec:symmetry_consequences_in_media}

The symmetry of each them plays a very important role in nonlinear optics. As
mentioned in the beginning of this section, $\boldsymbol{\chi}^{(2)}$ is a
third rank tensor, and it has 27 elements. The number of non-zero or
independent elements depends on the symmetry properties of the nonlinear media
in question. In crystallography, there are 32 crystal classes and we can find
two big groups, centrosymmetric and non-centrosymmetric. A centrosymmetric
media is a system with an inversion of symmetry center. It means that the
structure has a point $(x,y,z)$ in the unit cell that is equivalent at
$(-x,-y,-z)$. Then, if we are in a centrosymmetric media, the second order
nonlinear polarization can be written as Eq. \eqref{eq:second-pol-1}. If the
system presents inversion of symmetry, a inversion in the coordinates will
change both, the electric field and the polarization, since they are polar
vectors \cite{anderson16PHDT, jackson98book}, and so
\begin{equation}
\begin{aligned}
- \mathbf{P}^{(2)} (\mathbf{r},\omega) 
&= \epsilon_{0} \boldsymbol{\chi}^{(2)}[\mathbf{E}(-\mathbf{r},\omega)]^{2}
, \\
&= \epsilon_{0} \boldsymbol{\chi}^{(2)}[-\mathbf{E}(\mathbf{r},\omega)]^{2}
, \\
&= \epsilon_{0} \boldsymbol{\chi}^{(2)}[ \mathbf{E}(\mathbf{r},\omega)]^{2}
.
\end{aligned}
\end{equation}
So we have that $- \mathbf{P}^{(2)} (\mathbf{r},\omega) = \mathbf{P}^{(2)}
(\mathbf{r},\omega)$, and since we assume that $\mathbf{E} (\mathbf{r},\omega)
\neq 0$, then we conclude that
\begin{equation}
\boldsymbol{\chi}^{(2)} = 0
.
\end{equation}
In table \ref{tab:nonvanishing} we present the the form of the second order
susceptibility tensor for each of the 32 crystal classes with each element
denoted by Cartesian indices \cite{boyd03book}.

\begin{table}[H]
\centering
\caption{Form of the second order susceptibility tensor for each of the 32
crystal classes. Each element is denoted by its Cartesian indices.
\label{tab:nonvanishing}}
\begin{tabular}{llp{9cm}}
\hline
\hline
\textbf{Crystal system} & 
\textbf{Crystal class} & 
\textbf{Nonvanishing Tensor Elements} 
\\
\hline
Triclinic &
$1=C_{1}$ &
{All elements are independent and nonzero }
\\ &
$\overline{1}=S_{2}$
&
{Each element vanishes}
\\
\hline
Monoclinic & 
$2=C_{2}$ & 
$xyz$, $xzy$, $xxy$, $xyx$, $yxx$, $yyy$, $yzz$, $yzx$, $yxz$,
$zyz$, $zzy$, $zxy$, $zyx$ (twofold axis parallel to $\hat{y}$)
\\ &
$m=C_{1h} $ & 
$xxx$, $xyy$, $xzz$, $xzx$, $xxz$, $yyz$, $yzy$, $yxy$, $yyx$,
$zxx$, $zyy$, $zzz$, $zzx$, $zxz$ (mirror plane perpendicular to $\hat{y}$)
\\ &
$2/m=C_{2h}$ &
{Each element vanishes}
\\
\hline
Orthorhombic &
$222=D_{2}$ &
$xyz$, $xzy$, $yzx$, $yxz$, $zxy$, $zyx$
\\ &
$mm2=C_{2v}$ &
Each element vanishes
\\
\hline
Tetragonal &
$4=C_{4}$ & 
$xyz=-yxz$, $xzy=-yzx$, $xzx=yzy$, $xxz=yyz$, $zxx=zyy$, 
$zzz$, $zxy=-zyx$
\\ &
$\overline{4}=S_{4}$ &
$xyz=yxz$, $xzy=yzx$, $xzx=-yzy$, $xxz=-yyz$, $zxx=-zyy$, $zxy=zyx$
\\ &
$422=D_{4}$ &
$xyz=-yxz$, $xzy=-yzx$, $zxy=-zyx$
\\ &
$4mm = C_{4v} $ &
$xzx=yzy$, $xxz=yyz$, $zxx=zyy$, $zzz$
\\ &
$\overline{4}2m = D_{2d} $ &
$xyz=yxz$, $xzy=yzx$, $zxy=zyx$
\\ &
$4/m = C_{4h} $ &
Each element vanishes
\\ &
$4/mmm = D_{4h}$ &
Each element vanishes
\\
\hline
Cubic &
$432 = O$ &
$xyz=-xzy=yzx=-yxz=zxy=-zyx$
\\ &
$\overline{4}3 m = T_{d}$ &
$xyz=xzy=yzx=yxz=zxy=zyx$
\\ &
$23 = T$ &
$xyz=yzx=zxy$, $xzy=yxz=zyx$
\\ &
$m3 = T_{h}$, $m3m = O_{h}$ &
Each element vanishes
\\
\hline
Trigonal &
$3=C_{3}$ &
$xxx=-xyy=-yyz=-yxy$, $xyz=-yxz$, $xzy=-yzx$, $xzx=yzy$, $xxz=yyz$,
$yyy=-yxx=-xxy=-xyx$, $zxx=zyy$, $zzz$, $zxy=-zyx$
\\ &
$32=D_{3}$ &
$xxx=-xyy=-yyx=-yxy$, $xyz=-yxz$, $xzy=-yzx$, $zxy=-zyx$
\\ &
$3m = C_{3v}$ &
$xzx=yzy$, $xxz=yyz$, $zxx=zyy$, $zzz$, $yyy=-yxx=-xxy=-xyx$ (mirror plane
perpendicular to $\hat{x}$)
\\ &
$\overline{3}= S_{6}$, $\overline{3}m=D_{3d}$ &
Each element vanishes
\\
\hline
Hexagonal &
$6=C_{6}$ &
$xyz=-yxz$, $xzy=-yzx$, $xzx=yzy$, $xxz=yyz$, $zxx=zyy$, $zzz$, $zxy=-zyx$
\\ &
$\overline{6}=C_{3h}$ &
$xxx=-xyy=-yxy=-yyx$, $yyy=-yxx=-xyx=-xxy$
\\ &
$622=D_{6}$ &
$xyz=-yxz$, $xzy=-yxz$, $zxy=-zyx$
\\ &
$6mm = C_{6v}$ &
$xzx=yzy$, $xxz=yyz$, $zxx=zyy$, $zzz$
\\ &
$\overline{6}m2=D_{3h}$ &
$yyy=-yxx=-xxy=-xyx$
\\ &
$6/m = C_{6h}$ &
Each element vanishes
\\ &
$6/mmm = D_{6h}$ &
Each element vanishes
\\
\hline
\hline
\end{tabular}
\end{table}

\section{Graphene and Hydrogenated Graphene}
\label{sec:graphene_and_hydrogenated_graphene}

Graphene is an allotrope of carbon consisting of planar mono-atomic sheets of
$sp^{2}$ bonded carbon atoms. This synthetic structure is a densely packed 
two-dimensional equilateral triangular crystal lattice \cite{geim07Nat} with a
carbon-carbon chemical bond length of 0.142\,nm \cite{rao09JMC}. 
% 
In 2010 the Nobel Prize in Physics was awarded to Andre Geim and Konstantin
Novoselov for their breaking new work about graphene\footnote{Nobel Foundation
announcement of
\href{http://nobelprize.org/nobel_prizes/physics/laureates/2010/}{Nobel Price},
2010.}.

\subsection{Graphene properties}
\label{sec:graphene_properties}

Due to the properties that the structure presents, graphene became one of the
most relevant research topics in condensed matter physics and material science
for the last eight years \cite{geim07Nat, rao09JMC, li08NN,
katsnelson07MT, rao10PChL} since its first synthesis in 2004.

\stopcontents[chapters]

















